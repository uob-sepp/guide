%%%%%%%%%%%%%%%%%%%%%%%%%%%%%%%%%%%%%%%%%%%%%%%%%%%%%%%%%%%%%%%%%%%%%%%%%%%%%%%%
%% LaTeX sources for The Guide to Software Engineering & Professional Practice
%% Michael B. Gale (m.gale@bham.ac.uk)
%%
%% This work is licensed under the Creative Commons
%% Attribution-NonCommercial-NoDerivatives 4.0 International License. To
%% view a copy of this license, visit
%% http://creativecommons.org/licenses/by-nc-nd/4.0/ or send a letter to
%% Creative Commons, PO Box 1866, Mountain View, CA 94042, USA.
%%%%%%%%%%%%%%%%%%%%%%%%%%%%%%%%%%%%%%%%%%%%%%%%%%%%%%%%%%%%%%%%%%%%%%%%%%%%%%%%

\clearpage
\section{Coursework III: The Reflections}

Regardless of what career you choose to pursue after completing your degree, a good skill to have is to routinely reflect on the following:

\begin{itemize}
    \item The work you did since your last reflections.
    \item What your work accomplished.
    \item What new skills you acquired from doing the work.
    \item Where things could have gone better and what you would do differently if you did the same work again.
    \item What you will focus on accomplishing until the next reflections.
\end{itemize}

Doing this may help you understand what weaknesses you need to work on, what strengths you can utilise, and shape the direction your career progresses in. Some organisations may also employ a reflections process between an employee and their manager, either as part of a rewards process or to allow managers to support employees with their careers.

\subsection{Task}

Write a document of \emph{no more than two pages} reflecting on your participation in this module. You should answer all of the following questions:

\begin{enumerate}
    \item Think of your three biggest accomplishments during this module: describe what they are and how will they affect you going forward.
    \item Was there an occasion where something you did related to the module affected someone else positively or negatively? What did you learn from it?
    \item Think of one occasion where you did something that you would do differently now under the same circumstances.
    \item Identify one topic that we have covered or one skill you needed that you think is a particular strength of yours. How are you planning to make use of this strength going forward?
    \item Identify one topic that we have covered or one skill you needed that you think is a particular weakness of yours. How are you planning to work on it going forward?
\end{enumerate}

Remember that we are not looking for you to talk about how great everything has been. Instead, we are looking for honest and thoughtful reflections that provide a constructive assessment of yourself. You can gain full marks for this coursework even if things did not go so well, as long as you write about what happened, what you learnt from it, and what you will do differently in the future. On the other hand, you may not get many marks if you have nothing constructive to write about yourself.

\subsection{Assessment}

Your answer to each of the five questions carries a weight of 20\% for this coursework. Each answer will be marked according to the following scale:

\paragraph{20\%} An answer which shows an excellent level of critical reflection and is communicated clearly.

\paragraph{10\%} An answer which shows a reasonable level of critical reflection, or a reasonable answer that is not communicated clearly.

\paragraph{0\%} No answer, an answer which shows little evidence of critical reflection, or an incomprehensible answer.
