%%%%%%%%%%%%%%%%%%%%%%%%%%%%%%%%%%%%%%%%%%%%%%%%%%%%%%%%%%%%%%%%%%%%%%%%%%%%%%%%
%% LaTeX sources for The Guide to Software Engineering & Professional Practice
%% Michael B. Gale (m.gale@bham.ac.uk)
%%
%% This work is licensed under the Creative Commons
%% Attribution-NonCommercial-NoDerivatives 4.0 International License. To
%% view a copy of this license, visit
%% http://creativecommons.org/licenses/by-nc-nd/4.0/ or send a letter to
%% Creative Commons, PO Box 1866, Mountain View, CA 94042, USA.
%%%%%%%%%%%%%%%%%%%%%%%%%%%%%%%%%%%%%%%%%%%%%%%%%%%%%%%%%%%%%%%%%%%%%%%%%%%%%%%%

\clearpage
\section{Coursework III: The Reflections}

Regardless of what career you choose to pursue after completing your degree, a good skill to have is to routinely reflect on the following:

\begin{itemize}
    \item The work you did since your last reflections.
    \item What your work accomplished.
    \item What new skills you acquired from doing the work.
    \item Where things could have gone better and what you would do differently if you did the same work again.
    \item What you will focus on accomplishing until the next reflections.
\end{itemize}

Doing this may help you understand what weaknesses you need to work on, what strengths you can utilise, and shape the direction your career progresses in. Some organisations may also employ a reflections process between an employee and their manager, either as part of a rewards process or to allow managers to support employees with their careers.
