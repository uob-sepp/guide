%%%%%%%%%%%%%%%%%%%%%%%%%%%%%%%%%%%%%%%%%%%%%%%%%%%%%%%%%%%%%%%%%%%%%%%%%%%%%%%%
%% LaTeX sources for The Guide to Software Engineering & Professional Practice
%% Michael B. Gale (m.gale@bham.ac.uk)
%%
%% This work is licensed under the Creative Commons
%% Attribution-NonCommercial-NoDerivatives 4.0 International License. To
%% view a copy of this license, visit
%% http://creativecommons.org/licenses/by-nc-nd/4.0/ or send a letter to
%% Creative Commons, PO Box 1866, Mountain View, CA 94042, USA.
%%%%%%%%%%%%%%%%%%%%%%%%%%%%%%%%%%%%%%%%%%%%%%%%%%%%%%%%%%%%%%%%%%%%%%%%%%%%%%%%

\subsection{\higherEdFullName}

Universities depend on a large number of industry-specific software systems, such as for managing student records, detecting plagiarism, managing learning content, generating timetables, and so on. That is on top of extensive needs for more general-purpose software, such as office programs, single sign-on solutions, content management systems for websites, and so on.

\higherEdFullName\ is the market leader in providing comprehensive solutions for these software needs to universities.

Some of their existing products are outlined in the following sections.

\subsubsection{\higherEdCSRS}

\higherEdCSRS\ (pronounced \emph{courses}) is \higherEdShortName' core product offering. It combines student records, timetabling, and learning management, as well as many other features in one package. Universities can either host \higherEdCSRS\ on-premise in their own data centres, or \higherEdShortName\ offer \higherEdCSRS\ as a managed service that they host for universities. Since all universities are different, \higherEdCSRS\ provides many different configuration options that universities, departments, module organisers, and individual users can change. For more technically-savy users, \higherEdCSRS\ provides an API which can be used to automate many parts of \higherEdCSRS.

\subsubsection{\higherEdAntiCheat}

\higherEdAntiCheat\ is \higherEdShortName' state-of-the-art anti-plagiarism solution. It takes a coursework specification, produces variations of the specification using several, popular LLMs. For each LLMs variants of the coursework specification, as well as the original, each LLM is then asked to produce a solution to the specification. All of the resulting LLM-generated reports are then combined with the pool of student submissions in a traditional plagiarism checker. \higherEdAntiCheat\ is available to universities as a separate SaaS solution with a monthly or annual subscription. For universities that use \higherEdCSRS, \higherEdAntiCheat\ can easily be integrated with assignments.
