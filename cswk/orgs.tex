%%%%%%%%%%%%%%%%%%%%%%%%%%%%%%%%%%%%%%%%%%%%%%%%%%%%%%%%%%%%%%%%%%%%%%%%%%%%%%%%
%% LaTeX sources for The Guide to Software Engineering & Professional Practice
%% Michael B. Gale (m.gale@bham.ac.uk)
%%
%% This work is licensed under the Creative Commons
%% Attribution-NonCommercial-NoDerivatives 4.0 International License. To
%% view a copy of this license, visit
%% http://creativecommons.org/licenses/by-nc-nd/4.0/ or send a letter to
%% Creative Commons, PO Box 1866, Mountain View, CA 94042, USA.
%%%%%%%%%%%%%%%%%%%%%%%%%%%%%%%%%%%%%%%%%%%%%%%%%%%%%%%%%%%%%%%%%%%%%%%%%%%%%%%%

\section{Your organisation}
\label{sec:organisations}

To frame the tasks you will be carrying out for this module's assessments, you will choose one organisation from the following options. You will \emph{pretend} to work for this organisation and the assessments represent tasks that you are carrying out for that organisation. The choice is entirely yours, but you will stick with the same organisation \emph{for all three assessments}. In particular, note that the second coursework is carried out as a group. While you will be able to choose your own group, you can only team up with others \emph{in the same organisation}. You may wish to take this into consideration when choosing an organisation, as well as your own personal interests.

The descriptions of the companies and their products are merely supposed to give you an idea of what sort of companies they are, what sort of products they make, and how they broadly work. For your coursework, you are free to make further assumptions about other products not described here, features not described here that you rely on but exceed the scope of your work, or implementation aspects not mentioned here. You should however document any such assumptions you make.

%%%%%%%%%%%%%%%%%%%%%%%%%%%%%%%%%%%%%%%%%%%%%%%%%%%%%%%%%%%%%%%%%%%%%%%%%%%%%%%%
%% LaTeX sources for The Guide to Software Engineering & Professional Practice
%% Michael B. Gale (m.gale@bham.ac.uk)
%%
%% This work is licensed under the Creative Commons
%% Attribution-NonCommercial-NoDerivatives 4.0 International License. To
%% view a copy of this license, visit
%% http://creativecommons.org/licenses/by-nc-nd/4.0/ or send a letter to
%% Creative Commons, PO Box 1866, Mountain View, CA 94042, USA.
%%%%%%%%%%%%%%%%%%%%%%%%%%%%%%%%%%%%%%%%%%%%%%%%%%%%%%%%%%%%%%%%%%%%%%%%%%%%%%%%

\subsection{Student Smart Homes (SSH)}

Smart home technology is increasingly popular in homes throughout the UK. Student Smart Homes (SSH) is a start-up which specialises in producing hardware and software specifically aimed at student houses. The products are marketed either directly at students or providers of student accommodation. For students to be able to use the products, they must be suitable for use in rented accommodation. In other words, they must not require any permanent modifications to the accommodation. In either case, the products must be suitable for multiple users, who may not have the same level of trust for each other as members of the same family.

Some of their existing products are outlined in the following sections.

\subsubsection{The SSH Hub - First Class}

The SSH Hub - First Class is Student Smart Homes' most recent smart home hub product. The device is USB-powered and connects other SSH products in the student home (via Bluetooth or WiFi) to the SSH Cloud (via WiFi). It runs software which detect SSH products in range, communicates with them, and propagates information between the SSH Cloud and those devices.

\subsubsection{The SSH Cloud}

The SSH Cloud encompasses all of Student Smart Homes' online services. It includes both a web-based interface that users can access to manage their accounts and remote-control their devices, as well as services that exchange information with the SSH Hubs.

\subsubsection{The SSH Camera}

The SSH Camera is a smart camera which can be positioned almost anywhere thanks to versatile mounting options, including a roll of sticky tape that comes included for free in the box. The camera can be used for multiple different scenarios thanks to its AI features:

\begin{itemize}
    \item By pointing the SSH camera at the front door, it can detect who enters and leaves the student house. The SSH Console (sold separately) will then show a beautiful overview of who is or has been in the house and when.
    \item By pointing the SSH camera at the sink, it can detect who just dumps their dirty dishes in it and never washes up. The SSH Cloud (subscription required) will then automatically send notifications to the SSH App to remind those house mates to clean up.
\end{itemize}


%%%%%%%%%%%%%%%%%%%%%%%%%%%%%%%%%%%%%%%%%%%%%%%%%%%%%%%%%%%%%%%%%%%%%%%%%%%%%%%%
%% LaTeX sources for The Guide to Software Engineering & Professional Practice
%% Michael B. Gale (m.gale@bham.ac.uk)
%%
%% This work is licensed under the Creative Commons
%% Attribution-NonCommercial-NoDerivatives 4.0 International License. To
%% view a copy of this license, visit
%% http://creativecommons.org/licenses/by-nc-nd/4.0/ or send a letter to
%% Creative Commons, PO Box 1866, Mountain View, CA 94042, USA.
%%%%%%%%%%%%%%%%%%%%%%%%%%%%%%%%%%%%%%%%%%%%%%%%%%%%%%%%%%%%%%%%%%%%%%%%%%%%%%%%

\subsection{\higherEdFullName}

Universities depend on a large number of industry-specific software systems, such as for managing student records, detecting plagiarism, managing learning content, generating timetables, and so on. That is on top of extensive needs for more general-purpose software, such as office programs, single sign-on solutions, content management systems for websites, and so on.

\higherEdFullName\ is the market leader in providing comprehensive solutions for these software needs to universities.

Some of their existing products are outlined in the following sections.

\subsubsection{\higherEdCSRS}

\higherEdCSRS\ (pronounced \emph{courses}) is \higherEdShortName' core product offering. It combines student records, timetabling, and learning management, as well as many other features in one package. Universities can either host \higherEdCSRS\ on-premise in their own data centres, or \higherEdShortName\ offer \higherEdCSRS\ as a managed service that they host for universities. Since all universities are different, \higherEdCSRS\ provides many different configuration options that universities, departments, module organisers, and individual users can change. For more technically-savy users, \higherEdCSRS\ provides an API which can be used to automate many parts of \higherEdCSRS.

\subsubsection{\higherEdAntiCheat}

\higherEdAntiCheat\ is \higherEdShortName' state-of-the-art anti-plagiarism solution. It takes a coursework specification, produces variations of the specification using several, popular LLMs. For each LLMs variants of the coursework specification, as well as the original, each LLM is then asked to produce a solution to the specification. All of the resulting LLM-generated reports are then combined with the pool of student submissions in a traditional plagiarism checker. \higherEdAntiCheat\ is available to universities as a separate SaaS solution with a monthly or annual subscription. For universities that use \higherEdCSRS, \higherEdAntiCheat\ can easily be integrated with assignments.

