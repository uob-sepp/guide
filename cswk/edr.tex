%%%%%%%%%%%%%%%%%%%%%%%%%%%%%%%%%%%%%%%%%%%%%%%%%%%%%%%%%%%%%%%%%%%%%%%%%%%%%%%%
%% LaTeX sources for The Guide to Software Engineering & Professional Practice
%% Michael B. Gale (m.gale@bham.ac.uk)
%%
%% This work is licensed under the Creative Commons
%% Attribution-NonCommercial-NoDerivatives 4.0 International License. To
%% view a copy of this license, visit
%% http://creativecommons.org/licenses/by-nc-nd/4.0/ or send a letter to
%% Creative Commons, PO Box 1866, Mountain View, CA 94042, USA.
%%%%%%%%%%%%%%%%%%%%%%%%%%%%%%%%%%%%%%%%%%%%%%%%%%%%%%%%%%%%%%%%%%%%%%%%%%%%%%%%

\pagebreak
\section{Coursework I: The Engineering Design Review}

When joining an organisation, it is unlikely that you will start an entirely new piece of software from scratch. Instead, you will be contributing to an existing software system. Depending on the organisation you work for, the process by which work is planned and allocated to you may vary considerably. However, it is likely that the process is primarily driven by business needs in one way or another. Some examples of common processes include:

\begin{itemize}
    \item Your organisation's leadership decide what to work on, based on their thoughts on what will allow the organisation to make (more) money. Your management chain may then divide the necessary work up and allocate some to you.
    \item The leadership team has identified broad priorities for the organisation which product managers work to design the product around. Product managers will then coordinate with engineering managers to plan how the product they designed can be engineered, and some of the resulting work is then allocated to you.
    \item Some organisations may give total freedom to engineers to work on features that they consider to be important.
\end{itemize}

Regardless of the process used, software engineers like you will be required to come up with solutions for the features that are a business priority. For example, you may be told that some feature is deemed necessary and that you should work on it, but it may be up to you to figure out how exactly it should work and integrate with other parts of the software.

Where there is a design space -- in other words, there are multiple different options for how a given feature could be implemented -- it makes sense to describe the approach that you think is best in a document for an Engineering Design Review (EDR) and compare it with other alternatives that you considered. The EDR is an opportunity for other engineers on your team or other teams who would be affected by your implementation to give you feedback on the proposed approach.

Sometimes, EDRs can also be a means by which engineers can propose some feature that they think is worth working on. In that case, it is likely that engineering and product managers will also review the document to decide whether the feature is indeed worth allocating time for and fits in with the overall product direction.
