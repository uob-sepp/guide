%%%%%%%%%%%%%%%%%%%%%%%%%%%%%%%%%%%%%%%%%%%%%%%%%%%%%%%%%%%%%%%%%%%%%%%%%%%%%%%%
%% LaTeX sources for The Guide to Software Engineering & Professional Practice
%% Michael B. Gale (m.gale@bham.ac.uk)
%%
%% This work is licensed under the Creative Commons
%% Attribution-NonCommercial-NoDerivatives 4.0 International License. To
%% view a copy of this license, visit
%% http://creativecommons.org/licenses/by-nc-nd/4.0/ or send a letter to
%% Creative Commons, PO Box 1866, Mountain View, CA 94042, USA.
%%%%%%%%%%%%%%%%%%%%%%%%%%%%%%%%%%%%%%%%%%%%%%%%%%%%%%%%%%%%%%%%%%%%%%%%%%%%%%%%

\chapter{The Module}

For many of you, a degree in Computer Science will be the stepping stone to a career in software engineering. At the same time, most of your experience building software has been to write some Java code on your own, compile it directly with the Java compiler, and run it on your own machine. Real software engineering is very different.

In this module, we will explore the gap between the programming you will have done in your first year and the real software engineering you are likely to do in your careers. You will be introduced to the different contexts that software engineering takes place in and the different processes that organisations may employ surrounding it. Furthermore, you will gain foundational knowledge of relevant tools and technical skills.

This guide serves as a companion to the module by giving you an overview of all the major components. It will be a living document that will evolve to include more content as the module progresses.
