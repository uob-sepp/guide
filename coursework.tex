%%%%%%%%%%%%%%%%%%%%%%%%%%%%%%%%%%%%%%%%%%%%%%%%%%%%%%%%%%%%%%%%%%%%%%%%%%%%%%%%
%% LaTeX sources for The Guide to Software Engineering & Professional Practice
%% Michael B. Gale (m.gale@bham.ac.uk)
%%
%% This work is licensed under the Creative Commons
%% Attribution-NonCommercial-NoDerivatives 4.0 International License. To
%% view a copy of this license, visit
%% http://creativecommons.org/licenses/by-nc-nd/4.0/ or send a letter to
%% Creative Commons, PO Box 1866, Mountain View, CA 94042, USA.
%%%%%%%%%%%%%%%%%%%%%%%%%%%%%%%%%%%%%%%%%%%%%%%%%%%%%%%%%%%%%%%%%%%%%%%%%%%%%%%%

\section{Coursework}

There are three pieces of coursework which you will have to complete. You will receive feedback for the first coursework before the second and third coursework are due.

\subsection{The Engineering Design Review (40\%)}

\paragraph{Description} You get to write an \emph{Engineering Design Review} (EDR). An EDR is a document, written by a software engineer, which discusses a proposed design for new functionality in a software system. The document is primarily aimed at other engineers, but also engineering or product managers. An EDR is an opportunity to propose a solution to a problem, discuss your chosen design, explain why you chose it over alternatives, and propose how work on implementing it should be structured.

\paragraph{Aims} This coursework is designed to develop your ability to research a technical problem, propose a solution to it, think about the context in a wider project and organisation, and exercise your technical writing skills.

\subsection{The Prototype (40\%)}

\paragraph{Description} As a small group of 3-4, you will implement a \emph{prototype} based on \emph{one} of your EDRs from the first coursework. A prototype is an as-simple-as-possible implementation of your design that can demonstrate that your design is feasible. In other words, while the goal of your EDR is to describe a solution that makes sense to you and other engineers in theory, the goal of the prototype is to demonstrate that the design holds up once implemented, and no more.

\paragraph{Aims} This tests your ability to collaborate as a small team, choose appropriate implementation tools for your prototype, implement a design, and simulate interfaces and dependencies that your prototype relies on, but which do not actually exist.

\subsection{The Reflections (20\%)}

\paragraph{Description} On your own, you will write a short document reflecting on your participation in the module. By ``reflecting'', we mean ``thinking constructively about''. Reflecting on your development as an individual is a useful skill in any career path. Additionally, organisations may, in one way or another, also ask you to reflect on your development in the context of your employment. This can either be used to help you work towards your career goals, or factor into formal rewards conversations (conversations about salary increases, bonuses, etc.). In addition to reflecting about your own development and work, you may also reflect about colleagues, your team, the organisation you work for, etc.

\paragraph{Aims} Software engineering requires many different skills and, naturally, we are better at some than others. By reflecting on your work, you can figure out which skills you can leverage more strongly to benefit from, and which skills you need to work on going forward. This coursework tests your ability to think constructively about your work, development, and factors that contribute to it.
